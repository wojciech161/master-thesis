\pdfoutput=1
\pdfcompresslevel=9
\pdfinfo
{
    /Author ()
    /Title ()
    /Subject ()
    /Keywords ()
}

%\newcommand*{\memfontfamily}{pnc}
%\newcommand*{\memfontpack}{newcent}
\documentclass[a4paper,onecolumn,oneside,12pt]{memoir}
% Wydruk do archiwum
%\documentclass[a4paper,onecolumn,twoside,10pt]{memoir} 
%\renewcommand{\normalsize}{\fontsize{8pt}{10pt}\selectfont}
\usepackage[utf8]{inputenc}
\usepackage[T1]{fontenc}
\usepackage{setspace}
\usepackage{tabularx}
\usepackage{color,calc}
%\usepackage{soul} % pakiet z komendami do podkreślania tekstu
\usepackage{times} % pakiet zmieniający czcionki na times 

%\usepackage{longtable}
%\usepackage{ltxtable}
%\usepackage{tabulary}

\usepackage{indentfirst}
\usepackage{multicol}
\usepackage{listings}
\lstset{basicstyle=\footnotesize\ttfamily,breaklines=true}

%%%%%% Ustawienia odpowiedzialne za sposób łamania dokumentu
%\hyphenpenalty=10000		% nie dziel wyrazów zbyt często
\clubpenalty=10000      %kara za sierotki
\widowpenalty=10000  % nie pozostawiaj wdów
\brokenpenalty=10000		% nie dziel wyrazów między stronami
\exhyphenpenalty=999999		% nie dziel słów z myślnikiem
\righthyphenmin=3			% dziel minimum 3 litery

%\tolerance=4500
%\pretolerance=250
%\hfuzz=1.5pt
%\hbadness=1450

%ustawienia rozmiarów: tekstu, stopki, marginesów 
\setlength{\textwidth}{\paperwidth}
\addtolength{\textwidth}{-5cm}
\setlength{\textheight}{\paperheight}
\addtolength{\textheight}{-5cm}
\setlength{\oddsidemargin}{-0.04cm} % domyślnie jest 1 cal = 2.54 cm, stąd -0.04 da margines 2.5cm
\setlength{\evensidemargin}{-0.04cm} % domyślnie jest 1 cal = 2.54 cm, stąd -0.04 da margines 2.5cm
\topmargin -1.25cm 
\footskip 1.4cm 

\linespread{1}
%\linespread{1.3}

\usepackage{ifpdf}
%\newif\ifpdf \ifx\pdfoutput\undefined
%\pdffalse % we are not running PDFLaTeX
%\else
%\pdfoutput=1 % we are running PDFLaTeX
%\pdftrue \fi
\ifpdf
\usepackage[pdftex]{graphicx,hyperref}
\DeclareGraphicsExtensions{.pdf,.jpg,.mps,.png}
\pdfcompresslevel=9
\else
\usepackage{graphicx}
\DeclareGraphicsExtensions{.eps,.ps,.jpg,.mps,.png}
\fi
\sloppy

%\graphicspath{{rys01/}{rys02/}}
%\usepackage{rotating}

\renewcommand{\topfraction}{1.0}
\renewcommand{\bottomfraction}{1.0}
\renewcommand{\textfraction}{0.0}
\renewcommand{\floatpagefraction}{0.35}

%%%%%%%%%%%%%%%%%%%%%%%%%%%%%%%%%%%%%%%
%                  Definicja strony tytułowej 
%%%%%%%%%%%%%%%%%%%%%%%%%%%%%%%%%%%%%%%
\makeatletter
%Uczelnia
\newcommand\uczelnia[1]{\renewcommand\@uczelnia{#1}}
\newcommand\@uczelnia{}
%Wydział
\newcommand\wydzial[1]{\renewcommand\@wydzial{#1}}
\newcommand\@wydzial{}
%Kierunek
\newcommand\kierunek[1]{\renewcommand\@kierunek{#1}}
\newcommand\@kierunek{}
%Specjalność
\newcommand\specjalnosc[1]{\renewcommand\@specjalnosc{#1}}
\newcommand\@specjalnosc{}
%Tytuł po angielsku
\newcommand\titleEN[1]{\renewcommand\@titleEN{#1}}
\newcommand\@titleEN{}
%Tytuł krótki
\newcommand\titleShort[1]{\renewcommand\@titleShort{#1}}
\newcommand\@titleShort{}
%Promotor
\newcommand\promotor[1]{\renewcommand\@promotor{#1}}
\newcommand\@promotor{}

\def\maketitle{%
  \null
  \pagestyle{empty}%
	{\centering\vspace{-1cm}
		{\fontsize{22pt}{24pt}\selectfont \@uczelnia}\\[0.4cm]
		{\fontsize{22pt}{24pt}\selectfont \@wydzial }\\[0.5cm]
		\hrule \vspace*{0.7cm}
	}
{\flushleft\fontsize{14pt}{16pt}\selectfont%
\begin{tabular}{ll}
KIERUNEK: & \@kierunek\\
SPECJALNOŚĆ: & \@specjalnosc\\
\end{tabular}\\[1.3cm]
}
{\centering
\vskip 1cm
{\fontsize{24pt}{26pt}\selectfont PRACA DYPLOMOWA}\\[0.5cm]
{\fontsize{24pt}{26pt}\selectfont MAGISTERSKA}\\[2cm]
%{\fontsize{24pt}{26pt}\selectfont PROJEKT INżYNIERSKI}\\[1.5cm]
\vskip 0.8cm
}
%
\begin{tabularx}{\linewidth}{p{6cm}>{\centering\arraybackslash}X}
		&{\fontsize{16pt}{18pt}\selectfont \@title}\\[5mm] 	%UWAGA: tutaj jest miejsce na tytył w języku polskim
		&{\fontsize{16pt}{18pt}\selectfont \@titleEN}\\[10mm] %UWAGA: tutaj jest miejsce na tytył w języku angielskim
\end{tabularx}
\vfill
\begin{tabularx}{\linewidth}{p{6cm}l}
		%UWAGA: tutaj jest miejsce na autora pracy
		&{\fontsize{16pt}{18pt}\selectfont AUTOR:}\\[5mm]
		&{\fontsize{14pt}{16pt}\selectfont \@author}\\[10mm]
		%UWAGA: tutaj jest miejsce na promotora pracy 
		&{\fontsize{16pt}{18pt}\selectfont PROWADZĄCY PRACĘ:}\\[5mm]
		&{\fontsize{14pt}{16pt}\selectfont \@promotor}\\[10mm]
		&{\fontsize{16pt}{18pt}\selectfont OCENA PRACY:}\\[20mm]
	\end{tabularx}
\hrule\vspace*{0.3cm}
{\centering
%{\fontsize{24pt}{26pt}\selectfont PRACA DYPLOMOWA}\\[0.5cm]
%{\fontsize{24pt}{26pt}\selectfont MAGISTERSKA}\\[2cm]
{\fontsize{16pt}{18pt}\selectfont \@date}\\[0cm]
}
\normalfont
 \cleardoublepage
}
\makeatother
%%%%%%%%%%%%%%%%%%%%%%%%%%%%%%%%%%%%%%%
%                  Styl rozdziałów 
%%%%%%%%%%%%%%%%%%%%%%%%%%%%%%%%%%%%%%%
\setcounter{secnumdepth}{3}
\setcounter{tocdepth}{3}
%\definecolor{niceblue}{rgb}{.168,.234,.671}

%\AtBeginDocument{% 
%        \addto\captionspolish{% 
%        \renewcommand{\tablename}{Table}% 
%}%} 

%\AtBeginDocument{% 
%        \addto\captionspolish{% 
%        \renewcommand{\chaptername}{Rozdział}% 
%}} 

%\AtBeginDocument{% 
%        \addto\captionspolish{% 
%        \renewcommand{\figurename}{Image}% 
%}%}
%        \addto\captionspolish{% 
%        \renewcommand{\bibname}{Literature}% 
%}


%%%%%%%%%%%%%%%%%%%%%%%%%%%%%%%%%%%%% nowe itemize
\makeatletter
\renewenvironment{itemize}{
  \begin{list}{  
  \csname labelitem\romannumeral\the\@listdepth\endcsname}{
  \setlength{\leftmargin}{1em}
	\setlength{\topsep}{6pt}%
	\setlength{\partopsep}{0pt}%
	\setlength{\parskip}{0pt}%
	\setlength{\parsep}{0pt}%
	\setlength{\itemsep}{0pt}}
}{
  \end{list}
}

%%%%%%%%%%%%%%%%%%%%%%%%%%%%%%%%%%%%%%%%%%%%%%%%%%%%%%%%%%%%%%%%%%                  Styl wyliczenia (opis skrótów) 
%%%%%%%%%%%%%%%%%%%%%%%%%%%%%%%%%%%%%%%%%%%%%%%%%%%%%%%%%%%%%%%%%
\newenvironment{Ventry}[1]%
 {\begin{list}{}{\renewcommand{\makelabel}[1]{\textbf{##1}\hfill}%
   \settowidth{\labelwidth}{\textbf{#1}}%
   \setlength{\leftmargin}{3cm}}}%
 {\end{list}}

\addtopsmarks{headings}{%
\nouppercaseheads % added at the beginning
}{%
\createmark{chapter}{both}{shownumber}{}{. \space}
%\createmark{chapter}{left}{shownumber}{}{. \space}
\createmark{section}{right}{shownumber}{}{. \space}
}%use the new settings
\pagestyle{headings}

\newlength\mytemplengtha

\setcounter{secnumdepth}{2}
\setcounter{tocdepth}{2}
\setsecnumdepth{subsection} % activating subsubsec numbering in doc

\makeatletter
\copypagestyle{outer}{headings}
\makeoddhead{outer}{}{}{\slshape\rightmark}
\makeevenhead{outer}{\slshape\leftmark}{}{}
\makeoddfoot{outer}{\@author:~\@titleEN}{}{\thepage}
\makeevenfoot{outer}{\thepage}{}{\@author:~\@titleEN}
\makeheadrule{outer}{\linewidth}{\normalrulethickness}
\makefootrule{outer}{\linewidth}{\normalrulethickness}{6pt}
\makeatother

% fix plain
%\copypagestyle{plain}{outer} % overwrite plain with outer
\makeoddhead{plain}{}{}{} % remove right header
\makeevenhead{plain}{}{}{} % remove left header
\makeevenfoot{plain}{}{}{}
\makeoddfoot{plain}{}{}{}

%\copypagestyle{plain}{outer} % overwrite plain with outer
\makeoddhead{empty}{}{}{} % remove right header
\makeevenhead{empty}{}{}{} % remove left header
\makeevenfoot{empty}{}{}{}
\makeoddfoot{empty}{}{}{}

\pagestyle{outer}

%definicja nagłówków
%\renewcommand{\chaptermark}[1]{\markboth{\ifnum \c@secnumdepth >\z@ \thechapter \ \fi #1}{}} 
%\renewcommand{\chaptermark}[1]{\markboth{\thechapter \ #1}{\thesection \ #1}} 
%\renewcommand{\sectionmark}[1]{\markright{\thesection \ #1}{}} 
%\renewcommand{\chaptermark}[1]{\markboth{\thechapter \MakeUppercase{#1}}{}}

%kropki po numerach sekcji
\makeatletter
\def\@seccntformat#1{\csname the#1\endcsname.\quad}
\def\numberline#1{\hb@xt@\@tempdima{#1\if&#1&\else.\fi\hfil}}
\makeatother

\renewcommand{\chapternumberline}[1]{#1.\quad}
\renewcommand{\cftchapterdotsep}{\cftdotsep}

%\AtBeginDocument{\addtocontents{toc}{\protect\thispagestyle{empty}}}

%\includeonly{wstep,rozdzial01} % jeśli chcemy kompilować tylko fragmenty, to można tu je wpisać
%%%%%%%%%%%%%%%%%%%%%%%%%%%%%%%%%%%%%%%%%
%                  Początek dokumentu 
%%%%%%%%%%%%%%%%%%%%%%%%%%%%%%%%%%%%%%%%%
\begin{document}
\title{Filtracja i segmentacja drukowanych w kolorze obrazów}
\titleShort{Filtracja i segmentacja drukowanych w kolorze obrazów}
\titleEN{Preprocessing and segmentation of color-printed images}
\author{Marcin Wojciechowski}
\uczelnia{POLITECHNIKA WROCŁAWSKA}
\wydzial{WYDZIAŁ ELEKTRONIKI}
\kierunek{INFORMATYKA}
\specjalnosc{Internet Engineering}
\promotor{dr inż. Tomasz Babczyński}
\date{WROCŁAW, 2014}
\maketitle

\pagestyle{outer}
\mbox{}\pdfbookmark[0]{Table of Contents}{spisTresci.1}
\tableofcontents* 
\newpage

\chapter{Introduction}

Beginning of this chapter is focused on a problem background. Next part is a 
full description of the topic and definition of main objectives of this thesis.
Finally there is presented a full structure of this document.

\section{Background}

Nowadays image digitization is a very important topic. It has a lot of practical applications.
For example, it is a very good way to prevent destruction of old photos,
to archive them(to keep them in one place in a digital version) and make it easier to 
send it from one place to another. Unfortunately, basic method of digitization of
images - scanning - has many disadvantages. Due to errors of a person, who does the
scanning, input image distortions or quality of a scanning device, resulting digitized images 
have a poor quality. To improve digitization process there are many filtration algorithms,
which task is to solve this problem and to produce an image with better quality. There
is another big advantage of image digitization - there is a possibility
to retrieve information from the picture(for example: pattern recognition, feature extraction).
This feature is handled by segmentation algorithms. Although this group is one of the most 
difficult in image processing, there is a big emphasis on a research, due to many advantages
of this type of algorithms in practical use.

\section{Problem statement and main objectives}

The main aim of this thesis is to create a filtration and segmentation algorithm, which should
apply to scanned color-printed images. In my work I will focus on scanned maps. 
Resulting algorithm should be optimized to used it with this type of scanned images. 
Output image should have following properties:

\begin{itemize}
  \item most of distortions should be eliminated;
  \item image should be divided into consistent areas, like forests, lakes, rivers, etc. These areas should
        be marked on the map in a consistent way(it should have the same color);
  \item detected areas should have the same size like in an input image;
  \item small details, like text should be kept and marked in a resulting image.
\end{itemize}

Used fltering algorithms should remove main distortions and all disadvantages of image scanning(like for example
rasters, moro effect). Next, proper segmentation algorithm should be used to detect all areas in an input map
and to divide the map into different, homogenous regions. \\

The final solution - set of algorithms - should be impemented in a chosen programming language. Created application
should be tested on many different scanned maps. Test results should be then analysed.

\section{Document structure}

This document has been divided into seven chapters and the bibliography at the end.
Below there is a short description of a content for each chapter:

% TODO do przepisania po napisaniu całości pracy. Na razie mocno nieaktualne

\begin{itemize}
  \item In a first chapter there is a description and the background of the topic, 
        next there are listed main objectives of the work and a problem statement;
  \item In a second chapter there is an analysis of existing research work and solutions
        of the problem;
  \item In a third chapter there is a theoretical introduction to the problem. There are 
        findings after analysis of input images, problems which should be taken into consideration
        and algorithm properties, which should be applied;
  \item Fourth chapter focuses on research part of this thesis. There is a full description of
        filtration and segmentation algorithms used in this work;
  \item Fifth chapter focuses on engineering part of the thesis. There is a description of 
        all implementation details of the algorithm;
  \item In a sixth chapter there are presented results of all taken tests;
  \item In a seventh chapter there are conclusions of the thesis.
\end{itemize}

\chapter{Theoretical introduction}



\section{Digital image processing}

According to \cite{digitalImageProcessing},
a digital image can be defined as a two dimentional function f(x, y), where x and y denote spatial 
coordinates and f(x, y) is the intensity or gray level of the image at this point. Values of x, y
and f(x, y) should be finite, descrete quantities. Digital image processing refers to processing 
digital images by means of a digital computer. It includes many primitive operations like noise
reduction, contrast enhancement or image sharpening. It also involves more advanced algorithms like
image segmentation(partitioning an image into regions or objects), description of objects,
classification(recognition) of objects and even advanced analysis of an image. Digital image 
processing algorithms take an image as an input. Output depends on a type of the algorithm.
It can be new, processed image or group of attributes taken from an image. Digital image processing
algorithms are already used in many areas, for example to categorize images, modelling, visualization,
medical diagnostics, industry and many others.

\section{Image filtration}

\section{Image segmentation}

\chapter{State of Art}

This chapter focuses on actual research in topic of filtration and segmentation of scanned color images.
There is also a description of basic filtration and segmentation algorithms used in this work.

\section{Digital image processing}

\section{Processing color scanned maps}

\section{Conclusion}


\chapter{Research design}

\section{Input data description}

\section{Problems and restrictions}

\section{Main parts of the algorithm}

\section{Detection of areas}

\section{Detection of smaller elements}

\section{Merge of layers}

\section{Conclusion}


\chapter{Implementation of the algorithm}

\section{Introduction}

\section{Used tools}

\section{Application outline}

\section{Conclusion}


\chapter{Tests and Results}

\section{Testing data}

\section{Result analysis}

\section{Conclusion}


\chapter{Conclusion}

\newpage

\begin{thebibliography}{   }

	\bibitem{digitalImageProcessing}{Rafael C. Gonzalez, E. Woods, Digital Image Processing, Prentice Hall, 3rd Edition(2007)}

\end{thebibliography}

\end{document}
