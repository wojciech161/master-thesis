\pdfoutput=1
\pdfcompresslevel=9
\pdfinfo
{
    /Author ()
    /Title ()
    /Subject ()
    /Keywords ()
}

\batchmode
\nonstopmode

%\newcommand*{\memfontfamily}{pnc}
%\newcommand*{\memfontpack}{newcent}
\documentclass[a4paper,onecolumn,oneside,12pt]{memoir}
% Wydruk do archiwum
%\documentclass[a4paper,onecolumn,twoside,10pt]{memoir} 
%\renewcommand{\normalsize}{\fontsize{8pt}{10pt}\selectfont}
\usepackage[utf8]{inputenc}
\usepackage[T1]{fontenc}
\usepackage{setspace}
\usepackage{tabularx}
\usepackage{color,calc}
%\usepackage{soul} % pakiet z komendami do podkreślania tekstu
\usepackage{times} % pakiet zmieniający czcionki na times 

%\usepackage{longtable}
%\usepackage{ltxtable}
%\usepackage{tabulary}

\usepackage{indentfirst}
\usepackage{multicol}
\usepackage{listings}
\lstset{basicstyle=\footnotesize\ttfamily,breaklines=true}

%%%%%% Ustawienia odpowiedzialne za sposób łamania dokumentu
%\hyphenpenalty=10000		% nie dziel wyrazów zbyt często
\clubpenalty=10000      %kara za sierotki
\widowpenalty=10000  % nie pozostawiaj wdów
\brokenpenalty=10000		% nie dziel wyrazów między stronami
\exhyphenpenalty=999999		% nie dziel słów z myślnikiem
\righthyphenmin=3			% dziel minimum 3 litery

%\tolerance=4500
%\pretolerance=250
%\hfuzz=1.5pt
%\hbadness=1450

%ustawienia rozmiarów: tekstu, stopki, marginesów 
\setlength{\textwidth}{\paperwidth}
\addtolength{\textwidth}{-5cm}
\setlength{\textheight}{\paperheight}
\addtolength{\textheight}{-5cm}
\setlength{\oddsidemargin}{-0.04cm} % domyślnie jest 1 cal = 2.54 cm, stąd -0.04 da margines 2.5cm
\setlength{\evensidemargin}{-0.04cm} % domyślnie jest 1 cal = 2.54 cm, stąd -0.04 da margines 2.5cm
\topmargin -1.25cm 
\footskip 1.4cm 

\linespread{1}
%\linespread{1.3}

\usepackage{ifpdf}
%\newif\ifpdf \ifx\pdfoutput\undefined
%\pdffalse % we are not running PDFLaTeX
%\else
%\pdfoutput=1 % we are running PDFLaTeX
%\pdftrue \fi
\ifpdf
\usepackage[pdftex]{graphicx,hyperref}
\DeclareGraphicsExtensions{.pdf,.jpg,.mps,.png}
\pdfcompresslevel=9
\else
\usepackage{graphicx}
\DeclareGraphicsExtensions{.eps,.ps,.jpg,.mps,.png}
\fi
\sloppy

%\graphicspath{{rys01/}{rys02/}}
%\usepackage{rotating}

\renewcommand{\topfraction}{1.0}
\renewcommand{\bottomfraction}{1.0}
\renewcommand{\textfraction}{0.0}
\renewcommand{\floatpagefraction}{0.35}

%%%%%%%%%%%%%%%%%%%%%%%%%%%%%%%%%%%%%%%
%                  Definicja strony tytułowej 
%%%%%%%%%%%%%%%%%%%%%%%%%%%%%%%%%%%%%%%
\makeatletter
%Uczelnia
\newcommand\uczelnia[1]{\renewcommand\@uczelnia{#1}}
\newcommand\@uczelnia{}
%Wydział
\newcommand\wydzial[1]{\renewcommand\@wydzial{#1}}
\newcommand\@wydzial{}
%Kierunek
\newcommand\kierunek[1]{\renewcommand\@kierunek{#1}}
\newcommand\@kierunek{}
%Specjalność
\newcommand\specjalnosc[1]{\renewcommand\@specjalnosc{#1}}
\newcommand\@specjalnosc{}
%Tytuł po angielsku
\newcommand\titleEN[1]{\renewcommand\@titleEN{#1}}
\newcommand\@titleEN{}
%Tytuł krótki
\newcommand\titleShort[1]{\renewcommand\@titleShort{#1}}
\newcommand\@titleShort{}
%Promotor
\newcommand\promotor[1]{\renewcommand\@promotor{#1}}
\newcommand\@promotor{}

\def\maketitle{%
  \null
  \pagestyle{empty}%
	{\centering\vspace{-1cm}
		{\fontsize{22pt}{24pt}\selectfont \@uczelnia}\\[0.4cm]
		{\fontsize{22pt}{24pt}\selectfont \@wydzial }\\[0.5cm]
		\hrule \vspace*{0.7cm}
	}
{\flushleft\fontsize{14pt}{16pt}\selectfont%
\begin{tabular}{ll}
KIERUNEK: & \@kierunek\\
SPECJALNOŚĆ: & \@specjalnosc\\
\end{tabular}\\[1.3cm]
}
{\centering
\vskip 1cm
{\fontsize{24pt}{26pt}\selectfont PRACA DYPLOMOWA}\\[0.5cm]
{\fontsize{24pt}{26pt}\selectfont MAGISTERSKA}\\[2cm]
%{\fontsize{24pt}{26pt}\selectfont PROJEKT INżYNIERSKI}\\[1.5cm]
\vskip 0.8cm
}
%
\begin{tabularx}{\linewidth}{p{6cm}>{\centering\arraybackslash}X}
		&{\fontsize{16pt}{18pt}\selectfont \@title}\\[5mm] 	%UWAGA: tutaj jest miejsce na tytył w języku polskim
		&{\fontsize{16pt}{18pt}\selectfont \@titleEN}\\[10mm] %UWAGA: tutaj jest miejsce na tytył w języku angielskim
\end{tabularx}
\vfill
\begin{tabularx}{\linewidth}{p{6cm}l}
		%UWAGA: tutaj jest miejsce na autora pracy
		&{\fontsize{16pt}{18pt}\selectfont AUTOR:}\\[5mm]
		&{\fontsize{14pt}{16pt}\selectfont \@author}\\[10mm]
		%UWAGA: tutaj jest miejsce na promotora pracy 
		&{\fontsize{16pt}{18pt}\selectfont PROWADZĄCY PRACĘ:}\\[5mm]
		&{\fontsize{14pt}{16pt}\selectfont \@promotor}\\[10mm]
		&{\fontsize{16pt}{18pt}\selectfont OCENA PRACY:}\\[20mm]
	\end{tabularx}
\hrule\vspace*{0.3cm}
{\centering
%{\fontsize{24pt}{26pt}\selectfont PRACA DYPLOMOWA}\\[0.5cm]
%{\fontsize{24pt}{26pt}\selectfont MAGISTERSKA}\\[2cm]
{\fontsize{16pt}{18pt}\selectfont \@date}\\[0cm]
}
\normalfont
 \cleardoublepage
}
\makeatother
%%%%%%%%%%%%%%%%%%%%%%%%%%%%%%%%%%%%%%%
%                  Styl rozdziałów 
%%%%%%%%%%%%%%%%%%%%%%%%%%%%%%%%%%%%%%%
\setcounter{secnumdepth}{3}
\setcounter{tocdepth}{3}
%\definecolor{niceblue}{rgb}{.168,.234,.671}

%\AtBeginDocument{% 
%        \addto\captionspolish{% 
%        \renewcommand{\tablename}{Table}% 
%}%} 

%\AtBeginDocument{% 
%        \addto\captionspolish{% 
%        \renewcommand{\chaptername}{Rozdział}% 
%}} 

%\AtBeginDocument{% 
%        \addto\captionspolish{% 
%        \renewcommand{\figurename}{Image}% 
%}%}
%        \addto\captionspolish{% 
%        \renewcommand{\bibname}{Literature}% 
%}


%%%%%%%%%%%%%%%%%%%%%%%%%%%%%%%%%%%%% nowe itemize
\makeatletter
\renewenvironment{itemize}{
  \begin{list}{  
  \csname labelitem\romannumeral\the\@listdepth\endcsname}{
  \setlength{\leftmargin}{1em}
	\setlength{\topsep}{6pt}%
	\setlength{\partopsep}{0pt}%
	\setlength{\parskip}{0pt}%
	\setlength{\parsep}{0pt}%
	\setlength{\itemsep}{0pt}}
}{
  \end{list}
}

%%%%%%%%%%%%%%%%%%%%%%%%%%%%%%%%%%%%%%%%%%%%%%%%%%%%%%%%%%%%%%%%%% Styl wyliczenia (opis skrótów) 
%%%%%%%%%%%%%%%%%%%%%%%%%%%%%%%%%%%%%%%%%%%%%%%%%%%%%%%%%%%%%%%%%
\newenvironment{Ventry}[1]%
 {\begin{list}{}{\renewcommand{\makelabel}[1]{\textbf{##1}\hfill}%
   \settowidth{\labelwidth}{\textbf{#1}}%
   \setlength{\leftmargin}{3cm}}}%
 {\end{list}}

\addtopsmarks{headings}{%
\nouppercaseheads % added at the beginning
}{%
\createmark{chapter}{both}{shownumber}{}{. \space}
%\createmark{chapter}{left}{shownumber}{}{. \space}
\createmark{section}{right}{shownumber}{}{. \space}
}%use the new settings
\pagestyle{headings}

\newlength\mytemplengtha

\setcounter{secnumdepth}{2}
\setcounter{tocdepth}{2}
\setsecnumdepth{subsection} % activating subsubsec numbering in doc

\makeatletter
\copypagestyle{outer}{headings}
\makeoddhead{outer}{}{}{\slshape\rightmark}
\makeevenhead{outer}{\slshape\leftmark}{}{}
\makeoddfoot{outer}{\@author:~\@titleEN}{}{\thepage}
\makeevenfoot{outer}{\thepage}{}{\@author:~\@titleEN}
\makeheadrule{outer}{\linewidth}{\normalrulethickness}
\makefootrule{outer}{\linewidth}{\normalrulethickness}{6pt}
\makeatother

% fix plain
%\copypagestyle{plain}{outer} % overwrite plain with outer
\makeoddhead{plain}{}{}{} % remove right header
\makeevenhead{plain}{}{}{} % remove left header
\makeevenfoot{plain}{}{}{}
\makeoddfoot{plain}{}{}{}

%\copypagestyle{plain}{outer} % overwrite plain with outer
\makeoddhead{empty}{}{}{} % remove right header
\makeevenhead{empty}{}{}{} % remove left header
\makeevenfoot{empty}{}{}{}
\makeoddfoot{empty}{}{}{}

\pagestyle{outer}

%definicja nagłówków
%\renewcommand{\chaptermark}[1]{\markboth{\ifnum \c@secnumdepth >\z@ \thechapter \ \fi #1}{}} 
%\renewcommand{\chaptermark}[1]{\markboth{\thechapter \ #1}{\thesection \ #1}} 
%\renewcommand{\sectionmark}[1]{\markright{\thesection \ #1}{}} 
%\renewcommand{\chaptermark}[1]{\markboth{\thechapter \MakeUppercase{#1}}{}}

%kropki po numerach sekcji
\makeatletter
\def\@seccntformat#1{\csname the#1\endcsname.\quad}
\def\numberline#1{\hb@xt@\@tempdima{#1\if&#1&\else.\fi\hfil}}
\makeatother

\renewcommand{\chapternumberline}[1]{#1.\quad}
\renewcommand{\cftchapterdotsep}{\cftdotsep}

%\AtBeginDocument{\addtocontents{toc}{\protect\thispagestyle{empty}}}

%\includeonly{wstep,rozdzial01} % jeśli chcemy kompilować tylko fragmenty, to można tu je wpisać
%%%%%%%%%%%%%%%%%%%%%%%%%%%%%%%%%%%%%%%%%
%                  Początek dokumentu 
%%%%%%%%%%%%%%%%%%%%%%%%%%%%%%%%%%%%%%%%%
\begin{document}
\title{Filtracja i segmentacja drukowanych w kolorze obrazów}
\titleShort{Filtracja i segmentacja drukowanych w kolorze obrazów}
\titleEN{Preprocessing and segmentation of color-printed images}
\author{Marcin Wojciechowski}
\uczelnia{POLITECHNIKA WROCŁAWSKA}
\wydzial{WYDZIAŁ ELEKTRONIKI}
\kierunek{INFORMATYKA}
\specjalnosc{Internet Engineering}
\promotor{dr inż. Tomasz Babczyński}
\date{WROCŁAW, 2014}
\maketitle

\pagestyle{outer}
\mbox{}\pdfbookmark[0]{Table of Contents}{spisTresci.1}
\tableofcontents* 
\newpage

\chapter{Introduction}

Beginning of this chapter is focused on a problem background. Next part is a 
full description of the topic and definition of main objectives of this thesis.
Finally there is presented a full structure of this document.

\section{Background}

Nowadays image digitization is a very important topic. It has a lot of practical applications.
For example, it is a very good way to prevent destruction of old photos,
to archive them(to keep them in one place in a digital version) and make it easier to 
send it from one place to another. Unfortunately, basic method of digitization of
images - scanning - has some disadvantages. Due to errors made during a scanning procedure,
input image distortions or quality of a scanning device, resulting digitized images 
have a poor quality. To improve digitization process there are many filtration algorithms,
which task is to solve this problem and to produce an image with better quality.Image digitization 
has another big advantage - there is a possibility to retrieve information from the picture(for 
example: pattern recognition, feature extraction). This feature is handled by segmentation 
algorithms. Although this group is one of the most difficult in image processing, there is a big 
emphasis on a research, due to many advantages of this type of algorithms in practical use.

\section{Problem statement and main objectives}

The main aim of this thesis is to create a filtration and segmentation algorithm, which should
apply to scanned color-printed images. This work is focused on scanned maps. 
Resulting algorithm should be optimized to used it with this type of scanned images. 
Output image should have following properties:

\begin{itemize}
  \item most of distortions should be eliminated;
  \item image should be divided into consistent areas, like forests, lakes, rivers, etc. These areas
        should be marked on the map in a consistent way(it should have the same color);
  \item detected areas should have the same size like in an input image;
  \item small details, like text should be kept and marked in a resulting image.
\end{itemize}

Used fltering algorithms should remove main distortions and all disadvantages of image scanning(like
for example rasters, moro effect). Next, proper segmentation algorithm should be used to detect all
areas in an input map and to divide the map into different, homogenous regions. \\

The final solution - set of algorithms - should be impemented in a chosen programming language.
Created application should be tested on many different scanned maps. Test results should be then
analysed.

\section{Document structure}

This document has been divided into seven chapters and the bibliography at the end.
Below there is a short description of a content for each chapter:

% TODO do przepisania po napisaniu całości pracy. Na razie mocno nieaktualne

\begin{itemize}
  \item In a first chapter there is a description and the background of the topic, 
        next there are listed main objectives of the work and a problem statement;
  \item In a second chapter there is an analysis of existing research work and solutions
        of the problem;
  \item In a third chapter there is a theoretical introduction to the problem. There are 
        findings after analysis of input images, problems which should be taken into consideration
        and algorithm properties, which should be applied;
  \item Fourth chapter focuses on research part of this thesis. There is a full description of
        filtration and segmentation algorithms used in this work;
  \item Fifth chapter focuses on engineering part of the thesis. There is a description of 
        all implementation details of the algorithm;
  \item In a sixth chapter there are presented results of all taken tests;
  \item In a seventh chapter there are conclusions of the thesis.
\end{itemize}

\chapter{Theoretical introduction}

This section contains all needed definitions and concepts needed for a good understanding of the
work. It also covers history and basic usage of digital image processing. The last section is 
focused on actual state of art in the filtration and segmentation of color printed digital images.

\section{Digital image processing}

\subsection{History and usage}

According to \cite{digitalImageProcessing},
digital image processing has a lot of practical applications. It was used first at the beginning of
XX century to send pictures over long distances. Research and development of digital image
processing algorithms  was strongly associated with growth of the computer industry. It was caused
by high computational complexity of this type of algorithms. Further growth began in 1960s. Many 
techniques were developed in organisations like Jet Propulsion Laboratory, Massachusetts Institute
of Technology or Bell Laboratories. It was used in satellite imagery, medical imaging(in 1970s 
computerized tomography was developed), image enhancement, science(for example in geography - 
in research about pollution, archeology - in restoring blured pictures), defence or industry.
Another area of digital image processing focuses on extracting an information from the picture in
a for suitable for computer processing. It is used in a character recognition(OCR), industry(
automatic inspection of production process), military, forensics(recognition of fingerprints) and
weather prediction. \\

Nowadays, with fast computers and signal processors, digital image processing is used widely.
It is mainly caused by its low price and versatility.

\subsection{Basic definitions}

A digital image can be defined as a two dimentional function f(x, y), where x and y denote spatial 
coordinates and f(x, y) is the intensity or gray level of the image at this point. Values of x, y
and f(x, y) should be finite, descrete quantities. Digital image processing refers to processing 
digital images by means of a digital computer. It includes many primitive operations like noise
reduction, contrast enhancement or image sharpening. It also involves more advanced algorithms like
image segmentation(partitioning an image into regions or objects), description of objects,
classification(recognition) of objects and even advanced analysis of an image. Digital image 
processing algorithms take an image as an input. Output depends on a type of the algorithm.
It can be new, processed image or group of attributes taken from an image. Main tasks of digital
image processing are:

\begin{itemize}
  \item Image aquisition;
  \item Image enhancement;
  \item Image restoration;
  \item Wavelets;
  \item Compression;
  \item Feature extraction;
  \item Segmentation;
  \item Recognition.
\end{itemize}

This work focuses mainly on image enhancement and image segmentation. All algorithms used in this 
work are described in next sections.

\section{Image enhancement}

Image enhancement algorithms are one of the simplest and most commonly used group of digital image
processing. Main idea of these algorithms is to improve image quality, bring out obscured details 
or highlight certain features of an image.  The main aim of image enhancement is to process image 
and return an image more suitable, than input for a specific application. Proper image enhancement
algorithms should be chosen for different cases(algorithm solving one problem may be inadequate for
other problems). Image enhancement algorithms can be divided into two groups:

\begin{itemize}
  \item Spatial domain - manipulation of pixels of an image directly;
  \item Frequency domain - modification of Fourier transform of an image.
\end{itemize}

\subsection{Smoothing algorithms}

According to \cite{learningOpenCv}, smoothing is a very simple and frequently used operation. It is
used to reduce noise or camera artifacts. It is also used in an image resolution reduction. In this
work, there are two smoothing algorithms used.

\subsubsection{Gaussian blur}

Gaussian filter is one of the most frequently used filters. According to \cite{learningOpenCv}, 
gaussian blur is done by convolving each point in the input array with a Gaussian kernel and then
summing to produce the output array. Resulting image is a smooth blur. It is used to reduce noise
or image details and as a part of edge detection algorithms. Example result of a Gaussian Blur
filter is shown in a Fig.~\ref{gaussianBlurExample}.

\begin{figure}[ht]
\begin{center}
\includegraphics[scale=1.0]{images/GaussianBlurExample.jpg}
\caption{Gaussian Blur example. \\
Source: http://en.wikipedia.org/wiki/File:Halftone,\_Gaussian\_Blur.jpg}
\label{gaussianBlurExample}
\end{center}
\end{figure}

\subsubsection{Bilateral filter}

According to \cite{bilateralFilterWiki}, bilateral filter is a non-linear, edge-preserving and
noise reducing filter for images. The intensity value at each pixel in an image is replaced by a
weighted average of an intensity values from nearby pixels. In contrast to the Gaussian Blur, 
weights are based on the difference of intensity from the center pixel. This algorithm is commonly
used to prepare an input for segmenting algorithms. Example of a bilateral filter usage is presented
in a fig.~\ref{bilateralFilterExample}.

\begin{figure}[ht]
\begin{center}
\includegraphics[scale=0.6]{images/BilateralFilterExample.jpg}
\caption{Bilateral Filter example. \\
Source: http://www.planet-source-code.com/Upload\_PSC/ScreenShots/PIC20103161922506817.jpg}
\label{bilateralFilterExample}
\end{center}
\end{figure}

\subsection{Morphological transformations}

Morphological operations are used to remove noise, isolate or joining disparate elements
in an image and find intensity bumps or holes in an image to find image gradients. Basic 
morphological operations are called Dilation and Erosion.

\subsubsection{Dilation}

Dilation is a convolution of an image with a kernel. The kernel can have any shape or size and has
to have defined an anchor point. When kernel is scanned over the image, there is computed maximum
pixel value covered by the kernel. Then, the anchor point value is set to maximum value. Dilation
causes growth of all bright elements in an image. Example of dilation is presented in 
Fig.~\ref{dilationExample}

\begin{figure}[ht]
\begin{center}
\includegraphics[scale=0.6]{images/dilationExample.jpg}
\caption{Dilation example. \\
Source: http://www.mathworks.com/help/images/ref/referenceipart130.gif}
\label{dilationExample}
\end{center}
\end{figure}

\subsubsection{Erosion}

Erosion is a converse operation to dilation. Main difference is that it computes local minimum over
the area of the kernel. Erosion causes growth of all dark elements in an image. Example of erosion
is presented in Fig.~\ref{erosionExample}

\begin{figure}[ht]
\begin{center}
\includegraphics[scale=0.6]{images/erosionExample.jpg}
\caption{Erosion example. \\
Source: http://www.mathworks.com/help/images/ref/referenceipart128.gif}
\label{erosionExample}
\end{center}
\end{figure}

\subsection{Denoising}

Main goal of this type of algorithms is to eliminate noise from the image. In this work algorithm
called Non-local Means Denoising \cite{nonLocalMeansDenoising}. Example of its usage is presented in
a Fig.~\ref{denoisingExample}

\begin{figure}[ht]
\begin{center}
\includegraphics[scale=0.3]{images/denoisingExample.jpg}
\caption{Denoising example. \\
Source: Image generator in \cite{nonLocalMeansDenoising}}
\label{denoisingExample}
\end{center}
\end{figure}

\subsection{Unsharp mask}

Unsharp mask is applied to highlight or enhance some details in an image. According to 
\cite{unsharpMaskWiki}, this technique uses a blured or unsharp image to create mask of the original
image. The unsharped mask is then combined with the negative image. As a result there is an image
less blurry than the original. In algorithm used in this work, Gaussian blur is used and then its
result is compared to the original image. If the difference is greater than specified threshold
value, images are subtracted. Example of unsharp mask usage is presented in a 
Fig.~\ref{unsharpMaskExample}

\begin{figure}[ht]
\begin{center}
\includegraphics[scale=1.2]{images/unsharpMaskExample.jpg}
\caption{Unsharp mask example. \\
Source: http://upload.wikimedia.org/wikipedia/commons/4/43/Unsharped\_eye.jpg}
\label{unsharpMaskExample}
\end{center}
\end{figure}

\section{Image segmentation}

According to \cite{digitalImageProcessing}, segmentation algorithms partitions images into 
constituent parts or objects. They are generally based on two basic properties of intensity values:
discontinuity and similarity. In first case, the approach is to partition an image based on abrupt
changes in intensity(like for example edges in an image). In second case, the approach is based on
partitioning an image into regions, that are similar(for example they have similar color). There
are many segmentation algorithms available. Solution presented in this work is based on thresholding
approach.

\subsection{Tresholding}

Thresholding, thanks to its properties and simple implementation, is one of the most commonly used
segmentation algorithms. In the most basic form, having gray level of each pixel in an image we can
find a value of threshold, which will separate groups with different gray-level values. Then we 
set appropriate, different color value for pixels, which are above and below threshold value.

Let's assume, that:
\begin{itemize}
  \item T is a threshold value
  \item f(x, y) is a gray-level value in a point x, y
  \item g(x, y) is a thresholded pixel value
\end{itemize}

$$
g(x, y) = \left\{ \begin{array}{ll}
1 & \textrm{if $f(x, y) \geq T$}\\
0 & \textrm{if $f(x, y) < T$}\\
\end{array} \right.
$$

Example image with applied thresholding( with T = 130 ) is presented in a 
Fig.~\ref{thresholdingExample}.

\begin{figure}[ht]
\begin{center}
\includegraphics[scale=0.15]{images/thresholdingExample.jpg}
\caption{Thresholding example. \\
Source: own}
\label{thresholdingExample}
\end{center}
\end{figure}

\section{Color scanned maps processing}

Processing of color scanned maps is a very intersting topic. There are some research works focusing
on it. In this section there are short descriptions of found six algorithms, which focus on this
problem.

In \cite{semanticAnalysisAndRecognition}, authors treat alphanumerical signs, letters, points, lines
and areas separately. Image is vectorized and segmented in separate processes. To separate 
alphanumerical characters, there is used a \textit{False Color Technique}. Next, the 
\textit{Composite Image Technique} is used to decompose image and link objects by their associated
names. Finally, set of separate modules grouped as \textit{A2R2V} detect alphanumeric, punctual and
linear characters in the map. In this algorithm there are used such techniques like thresholding,
color segmentation and neural networks.

Another approach is presented in \cite{comparativeAnalysisOfScannedMaps}. This work focused on 
decomposition of scanned topographic maps. Algorithm finds different areas(like lakes, rivers,
contours, text, grid and other) using map analysis and thresholding in a black and white mode.
Next, there are used filtering algorithms to remove elements smaller than expected and noise.
Then map texture is analysed. This step is helpful in detection of areas like for example lakes.
Algorithm uses thresholding, median filter and many color segmentation techniques.

Next algorithm is shown in \cite{colorsOfThePast}.

Another solution is presented in \cite{automaticVectorization}.

Next algorithm can be found in \cite{topographicMapsAutomaticVectorization}.

Last presented approach is shown in \cite{colorMapSegmentation}.

\chapter{Research design}

In this chapter there is a full description of an algorithm processing scanned maps. Additionaly,
there is a description of an input data and found problems related to it.

\section{Input data description}

Algorithm has two inputs. One is a file containing an image with a map. Second input is a
configuration file, which contains parameters of an algorithm.

\subsection{Map image file}

Input of the algorithm is an digital image, stored in a disk in one of many available formats, like
png, tif, jpg or bmp. Image should be a color-scanned map at any resolution. Exampke image is 
presented in a Fig.~\ref{inputExample}. It is a scanned color map of a Polish city - Brzeg.

\begin{figure}[!ht]
\begin{center}
\includegraphics[scale=0.78]{images/brzeg.jpg}
\caption{Example input.}
\label{inputExample}
\end{center}
\end{figure}

To confirm proper working of the algorithm, there were nine maps prepared. All input map files
with its description are listed in a table~\ref{tableInputFiles}. 

\begin{table}[!ht]
\begin{center}
\caption{Input files.}
\label{tableInputFiles}
\begin{tabular}{|c|c|c|c|}
  \hline
  Filename & Format & Resolution & Description \\
  \hline
  brzeg & tif & 3391x4671 & Map of Brzeg. It has text and multiple areas like \\
        &     &           & forests, lakes, roads and cities \\
  \hline
  bydgoszcz & bmp & 1747x1272 & Map of Bydgoszcz. Small number of areas, \\
            &     &           & high density of roads and small details \\
  \hline
  dolny\_slask & png & 5936x8669 & Old map of a Lower Silesia. No backround. \\
              &     &           & Many thin lines and rivers. \\
  \hline
  lubsza & tif & 3391x4671 & Map of surroundings of Lubsza. It has many areas \\
         &     &           & like forests and lakes. Image has a big damage. \\
         &     &           & Map was bent during scanning process. \\
  \hline
  masyw\_slezy & tif & 3883x4606 & Topographic map of Ślęża Mountain. It contains \\
              &     &           & many areas in different colors, roads, text and \\
              &     &           & topographic lines. \\
  \hline
  namyslow & jpg & 4814x5628 & Topographic map of Namysłów. It contains many \\
           &     &           & small details, topographic lines, small text  \\
           &     &           & and bright background of areas\\
  \hline
  pokoj & tif & 2921x2989 & Map of Pokój. It has text, multiple areas \\
        &     &           & (forests, lakes) and roads. \\
  \hline
  sleza & jpg & 1812x2756 & Map of Ślęża. It contains areas like forests, \\
        &     &           & lakes and roads. \\
  \hline
  sobotka & png & 2798x1824 & Map of Sobótka. It has many different areas, like \\
          &     &           & small details, text and roads. \\
  \hline
\end{tabular}
\end{center}
\end{table}

\subsection{Configuration file}

Created algorithm has number of parameters, which have to be determined. Application loads them
using special XML file. Parameters in a configuration file are divided to two layers. Meaning
and usage of each parameter is explained in next sections. Example XML configuration file is
listed in~\ref{listingConfig}

\begin{lstlisting}[caption=Example configuration XML file,label=listingConfig,language=xml]
<?xml version="1.0" encoding="UTF-8"?>
<configuration>
    <background_detection>
        <enabled>true</enabled>
        <gaussian_blur_radius>9</gaussian_blur_radius>
        <gaussian_blur_standard_deviation>0</gaussian_blur_standard_deviation>
        <diameter>20</diameter>
        <sigma_color>150</sigma_color>
        <sigma_space>150</sigma_space>
        <dilation_and_erosion_size>2</dilation_and_erosion_size>
        <dilation_and_erosion_counter>4</dilation_and_erosion_counter>
        <window_width>15</window_width>
        <max_number_of_colors>30</max_number_of_colors>
    </background_detection>
    <line_detection>
        <enabled>true</enabled>
        <threshold_value>100</threshold_value>
        <unsharp_mask_standard_deviation>5</unsharp_mask_standard_deviation>
        <denoising_factor>35</denoising_factor>
        <bilateral_filter_counter>5</bilateral_filter_counter>
        <gaussian_blur_radius>5</gaussian_blur_radius>
        <gaussian_blur_standard_deviation>0</gaussian_blur_standard_deviation>
        <gaussian_blur_counter>2</gaussian_blur_counter>
        <initial_threshold>10</initial_threshold>
        <mimimal_color_area>500</mimimal_color_area>
        <max_number_of_colors>15</max_number_of_colors>
        <diameter>20</diameter>
        <sigma_color>75</sigma_color>
        <sigma_space>50</sigma_space>
    </line_detection>
</configuration>
\end{lstlisting}

\section{Problems and restrictions}

There are a lot of problems associated with segmentation of scanned maps. Due to imperfection of
a scanning process, resulting color scanned digital images have many distortions.
There are also problems connected with input image(it can be damaged before scanning).
Moreover, raster derived from printed document should be removed.
Additionaly maps have many small elements and text. These details, as well as shape of background
areas (like for example lakes and forests) should be kept in resulting image.
All found problems are described more precisely in next sections.

\subsection{Input image and scanning process quality}

Scanned digital images have many distortions. They are related with damage of an input image, like
places, where maps are creased(example in Fig.~\ref{creaseExample}) or dirty(shown in a 
Fig.~\ref{dirtyExample}). It results in the existence of darker elements in the output image.

\begin{figure}[!ht]
\begin{center}
\includegraphics[scale=1.1]{images/creaseExample.jpg}
\caption{Example of distortion caused by creased maps.}
\label{creaseExample}
\end{center}
\end{figure}

\newpage

\begin{figure}[!ht]
\begin{center}
\includegraphics[scale=3.0]{images/dirtyExample.jpg}
\caption{Example of distortion caused by dirty map.}
\label{dirtyExample}
\end{center}
\end{figure}

Another problems are related to scanning process. Resulting images are noisy. Additionaly a
scanning device or scanned map can be dusty. These problems results in small distortions, which
should not be placed in an output image. Example of this situation is presented in a 
Fig.~\ref{dustExample}. It is marked in red rectangle.

\begin{figure}[!ht]
\begin{center}
\includegraphics[scale=3.0]{images/dustExample.jpg}
\caption{Example of distortion caused by dust.}
\label{dustExample}
\end{center}
\end{figure}

Finally, resulting image can have undesirable areas with wrong(usually darker) color. Their
existence is caused by mistakes made during scanning process, for example when image is not properly
put in a scanner surface. This situation is visualized in a Fig.~\ref{badScanExample}.

\begin{figure}[!ht]
\begin{center}
\includegraphics[scale=1.3]{images/badScanExample.jpg}
\caption{Example of distortion caused by improper scanning.}
\label{badScanExample}
\end{center}
\end{figure}

\subsection{Small details}

Input image contains many small details. They exist especially in city and topographic maps.
City maps contain a lot of small elements, like churches, parking places or gas stations.
Additionaly, in cities there are many roads and areas, which are marked in different colors. 
Furthermore they contain many small letters, which, for example, indicate names of streets, parks
or major buildings and places. Finally, in the city map, lines, roads and test are 
often crossing each other. Example fragment of a city map fragment is shown in a 
Fig.~\ref{cityExample}.

\begin{figure}[!ht]
\begin{center}
\includegraphics[scale=0.65]{images/cityExample.jpg}
\caption{Fragment of a map with a city.}
\label{cityExample}
\end{center}
\end{figure}

Another problems are in topographic maps. They have small letters and elements in different
colors. Additionaly they contain very thin topographic maps. Many of them are not continuous
(some lines are even marked as dots in the same color). These problems are shown in a fragment of a
topographic map in a Fig.~\ref{topographicMapExample}.

\begin{figure}[!ht]
\begin{center}
\includegraphics[scale=2.0]{images/topographicMapExample.jpg}
\caption{Fragment of a topographic map.}
\label{topographicMapExample}
\end{center}
\end{figure}

\subsection{Raster elimination}

Digital image coming from a scanner is a raster image. Every pixel is defined itself. This
representation has some disadvantages. One of them is called moro effect. It interferes with
segmentation algorithm. Example of a moro effect is shown in a Fig.~\ref{moroEffectExample}

\begin{figure}[!ht]
\begin{center}
\includegraphics[scale=11.0]{images/moroEffectExample.jpg}
\caption{Example of a moro effect.}
\label{moroEffectExample}
\end{center}
\end{figure}

\section{Main parts of the algorithm}

\section{Detection of areas}

\section{Detection of smaller elements}

\section{Merge of layers}

\chapter{Implementation of the algorithm}

\section{Used tools}

\section{Application outline}

\chapter{Tests and Results}

\section{Test description}

\section{Result analysis}

\chapter{Conclusion}

\newpage

\renewcommand\bibname{References}

\begin{thebibliography}{   }

	\bibitem{digitalImageProcessing}
          {Rafael C. Gonzalez, Richard E. Woods, Digital Image Processing, Prentice Hall, 3rd Edition, 2007}
  \bibitem{learningOpenCv}
          {Gary Bradski, Adrian Kaehler, Learning OpenCV. Computer Vision with the OpenCV Library,
          O`Reilly Media Inc., 2008}
  \bibitem{bilateralFilterWiki}
          {http://en.wikipedia.org/wiki/Bilateral\_filter, Version from 29.05.2014}
  \bibitem{unsharpMaskWiki}
          {http://en.wikipedia.org/wiki/Unsharp\_masking, Version from 1.06.2014}
  \bibitem{nonLocalMeansDenoising}
          {http://www.ipol.im/pub/art/2011/bcm\_nlm/, Version from 1.06.2014}
  \bibitem{semanticAnalysisAndRecognition}
          {Serguei Levachkine, Aurelio Velázquez, Victor Alexandrov, Mikhail Kharinov, 
          Semantic Analysis and Recognition of Raster-Scanned Color Cartographic Images,
          4th International Workshop, GREC 2001 Kingston, Ontario, Canada, September 7–8, 
          2001 Selected Papers, 2002}
  \bibitem{comparativeAnalysisOfScannedMaps}
          {C. Armenakis, F. Leduca, I. Cyra, F. Savopola, F. Cavayasb,
          A comparative analysis of scanned maps and imagery for mapping applications,
          ISPRS Journal of Photogrammetry and Remote Sensing Volume 57, Issues 5–6, April 2003,
          Pages 304–314}
  \bibitem{colorsOfThePast}
          {Stefan Leyk, Ruedi Boesch, Colors of the past: color image segmentation in historical 
          topographic maps based on homogeneity, GeoInformatica, January 2010, Volume 14, Issue 1, pp 1-21 }
  \bibitem{automaticVectorization}
          {Huali Zheng, Xianzhong Zhou, Hongbo Wang, Automatic Vectorization of Contour Lines Based
          on Deformable Model with Adaptive Flow Orientation, Proceedings of the 2003 IEEE
          International Conference on Robotics,Intelligent Systems and Signal Processing
          Changsha, China - October 2003}
  \bibitem{topographicMapsAutomaticVectorization}
          {Marc Pierrot Deseilligny, Robert Mariani, Jacque Labiche, Remy Mullot,
          Topographic Maps Automatic Interpretation: Some Proposed Strategies, Second International
          Workshop, GREC' 97 Nancy, France, August 22–23, 1997}
  \bibitem{colorMapSegmentation}
          {Alireza Khotanzad, Edd Zink, Color paper map segmentation using eigenvector line-fitting,
          Image Analysis and Interpretation, 1996., Proceedings of the IEEE Southwest Symposium on
          8-9 April 1996}

\end{thebibliography}

\end{document}
